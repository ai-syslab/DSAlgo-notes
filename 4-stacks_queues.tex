\documentclass{article}
\usepackage{amsmath}
\usepackage{listings}
\usepackage{xcolor}
\usepackage{graphicx}
\usepackage{geometry}
\usepackage{hyperref}
\geometry{margin=1in}

\title{Stacks and Queues}
\author{}
\date{}

\definecolor{codegray}{rgb}{0.5,0.5,0.5}
\definecolor{backcolour}{rgb}{0.95,0.95,0.92}

\lstdefinestyle{cppstyle}{
  backgroundcolor=\color{backcolour},
  commentstyle=\color{codegray},
  keywordstyle=\color{blue},
  numberstyle=\tiny\color{codegray},
  stringstyle=\color{red},
  basicstyle=\ttfamily\footnotesize,
  breakatwhitespace=false,
  breaklines=true,
  captionpos=b,
  keepspaces=true,
  numbers=none,
  numbersep=5pt,
  showspaces=false,
  showstringspaces=false,
  showtabs=false,
  tabsize=2,
  language=C++
}

\begin{document}

\maketitle


\section{Stacks}
A \textbf{stack} is a Last-In-First-Out (LIFO) data structure. Elements are inserted and removed from the same end, called the \textit{top}.

\begin{figure}[h]
  \centering
  \includegraphics[width=0.8\textwidth]{figs/stacks.png}
  \caption{Stack operation (from Wikimedia)}
  \label{fig:stack}
\end{figure}

\textbf{Linked List Implementation:}
Each node points to the next element. Insertion and deletion happen at the head of the list.

\begin{lstlisting}[style=cppstyle]
struct Node {
  int data;
  Node* next;
};

class Stack {
  Node* top;
public:
  Stack() : top(nullptr) {}
  void push(int x) {
    Node* n = new Node{x, top};
    top = n;
  }
  void pop() {
    if (top) {
      Node* temp = top;
      top = top->next;
      delete temp;
    }
  }
  int peek() const { return top->data; }
  bool empty() const { return top == nullptr; }
};
\end{lstlisting}

\textbf{STL: \texttt{std::stack}} is implemented using \texttt{std::deque} by default. It supports \texttt{push()}, \texttt{pop()}, \texttt{top()}, and \texttt{empty()}.

\begin{itemize}
  \item \textbf{Read:} $\mathcal{O}(1)$ for top element
  \item \textbf{Insert:} $\mathcal{O}(1)$ at head
  \item \textbf{Delete:} $\mathcal{O}(1)$ at head
  \item \textbf{Update:} $\mathcal{O}(1)$ if on top; $\mathcal{O}(n)$ otherwise
\end{itemize}

\section{Queues}
A \textbf{queue} is a First-In-First-Out (FIFO) data structure. Insertion occurs at the \textit{rear}, deletion at the \textit{front}.

\begin{figure}[h]
  \centering
  \includegraphics[width=0.8\textwidth]{figs/queues.png}
  \caption{Queue operation (from Wikimedia)}
  \label{fig:queue}
\end{figure}

\textbf{Linked List Implementation:}
We maintain both front and rear pointers.

\begin{lstlisting}[style=cppstyle]
struct Node {
  int data;
  Node* next;
};

class Queue {
  Node* front;
  Node* rear;
public:
  Queue() : front(nullptr), rear(nullptr) {}
  void enqueue(int x) {
    Node* n = new Node{x, nullptr};
    if (rear) rear->next = n;
    else front = n;
    rear = n;
  }
  void dequeue() {
    if (front) {
      Node* temp = front;
      front = front->next;
      if (!front) rear = nullptr;
      delete temp;
    }
  }
  int peek() const { return front->data; }
  bool empty() const { return front == nullptr; }
};
\end{lstlisting}

\textbf{STL: \texttt{std::queue}} uses \texttt{std::deque} internally. \texttt{std::deque} (double-ended queue) supports insertion/removal at both ends in $\mathcal{O}(1)$ time.

\begin{itemize}
  \item \textbf{Read:} $\mathcal{O}(1)$ at front
  \item \textbf{Insert:} $\mathcal{O}(1)$ at rear
  \item \textbf{Delete:} $\mathcal{O}(1)$ at front
  \item \textbf{Update:} $\mathcal{O}(n)$
\end{itemize}

\section{Note on \texttt{std::deque}}
A \texttt{deque} (double-ended queue) allows fast insertions/removals at both front and back. STL uses it as the default underlying container for both \texttt{std::stack} and \texttt{std::queue} for performance.

\section{Practice problem}
Leetcode Problem 20: \textit{Valid paranthesis}

\url{https://leetcode.com/problems/valid-parentheses/description/}


\end{document}